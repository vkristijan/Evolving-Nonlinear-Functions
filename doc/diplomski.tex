\documentclass[times, utf8, diplomski, numeric]{fer}
\usepackage{booktabs}
\usepackage{indentfirst}
\usepackage{hyperref} 

\begin{document}

\thesisnumber{2022}

\title{Pronalaženje Booleovih funkcija maksimalne nelinearnosti evolucijskim računanjem}
\author{Kristijan Vulinović}

\maketitle

\izvornik

\zahvala{}

\tableofcontents

\chapter{Uvod}
Kriptografija je grana računarske znanosti, bez čijih dostignuća bi današnji svakodnevni život izgledao nezamislivo drugačije.
Većina se ljudi svakodnevno služi kriptografijom, makar toga nisu nužno ni svjesni.
Bilo to korištenjem online bankarstva, online trgovine ili jednostavno bilo kojeg online servisa temeljenog na \textit{HTTPS} protokolu.
Navedeni primjeri često se služe asimetričnim kriptografskim algoritmima kako bi klijent i poslužitelj razmijenili ključeve koje potom koriste u simetričnim kriptografskim algoritmima kojima se enkriptiraju njihove poruke.

Booleove funkcije predstavljaju jedan od sastavnih elemenata kriptografskih algoritama, poput na primjer S-kutija kod algoritma \textit{DES} i \textit{AES} \cite{daemen1999aes}.
Kako bi se postigla što veća sigurnost takvih sustava, potrebno je pomno odabrati korištene funkcije, ovisno o određenim svojstvima koje one posjeduju.
Jedan od najpoznatijih napada je napad linearnom kriptoanalizom, prvi puta opisan u \cite{golic1994linear}.
Kako bi se postigla veća otpornost na tu vrstu napada, od posebnog je značaja svojstvo nelinearnosti Booleove funkcije.
Pored toga, za primjenu u kriptografskim algoritmima dodatno je važno i svojstvo balansiranosti.
Osim navedenog, važna su i brojna druga svojstva poput svojstva korelacijske otpornosti \engl{correlation immunity}, autokorelacije \engl{autocorrelation}, algebarski stupanj \engl{algebraic degree}, algebarske otpornosti \engl{algebraic immunity} i drugih opisanih u \cite{CryptographicBooleanFunctions}.

Primarni fokus ovog rada je pronalazak Booleovih funkcija maksimalne nelinearnosti, kao i balansiranih Booleovih funkcija maksimalne nelinearnosti, pritom ne razmatrajući ostala svojstva.
Posebna pozornost posvećena je balansiranim Booleovim funkcijama sa $8$ varijabli, koje se koriste u algoritmu \textit{Achterbahn} \cite{gammel2005achterbahn}, kao i u algoritmu \textit{RAKAPOSHI} \cite{cid2009rakaposhi}, za koje je pitanje maksimalne moguće nelinearnosti još uvijek otvoreno.
Naime, najniža gornja granica nelinearnosti navedenih funkcija iznosi $118$, dok je najviša do sad pronađena nelinearnost u iznosu $116$.


\chapter{Zaključak}

\bibliography{literatura}
\bibliographystyle{fer}

\begin{sazetak}
Booleove funkcije sastavni su element kriptografskih algoritama.
Kako bi se povećala otpornost na napade linearnom kriptoanalizom, od posebnog je značaja svojstvo nelinearnosti Booleove funkcije.
Booleove funkcije zadanog broja varijabli i maksimalne nelinearnosti nazivaju se Bent-funkcije, dok su sa stajališta primjene u kriptografskim algoritmima od posebnog interesa Booleove funkcije koje dodatno imaju i svojstvo balansiranosti.

U okviru ovog diplomskog rada proučeni su heuristički pristupi pronalaženja Booleovih funkcija maksimalne nelinearnosti te balansiranih Booleovih funkcija maksimalne nelinearnosti.
Implementirani su i međusobno uspoređeni optimizacijski postupci temeljeni na simuliranom kaljenju, genetskom algoritmu i genetskom programiranju.
Uspoređeni su i različiti načini prikazivanja Booleovih funkcija, poput zapisa u obliku tablice istinitosti te algebarskog zapisa, kao i razne funkcije izračuna dobrote rješenja u evolucijskim algoritmima.
Dodatno je predložen i analiziran novi način pretrage traženih funkcija, temeljen na analizi Walsh koeficijenata funkcije.

\kljucnerijeci{Booleove funkcije, nelinearnost, heuristička optimizacija, evolucijski algoritmi}
\end{sazetak}

\engtitle{Evolutionary Computation Based Search for Maximal Nonlinearity Boolean Functions}
\begin{abstract}
Boolean functions represent a crucial element for designing cryptographic algorithms.
In order to resist against linear cryptanalysis attack, it is essential for Boolean functions to have high nonlinearity.
Bent functions are Boolean functions with maximal possible nonlinearity for a given number of variables.
For the usage in cryptographic algorithms is is additionally important for functions to be balanced. 

This thesis is based upon researching heuristic search methods for maximal nonlinear Boolean functions and maximal nonlinear balanced Boolean functions.
Solutions based on simulated annealing, genetic algorithms and genetic programming are implemented and compared.
The impact of different function representations, such as truth tables and algebraic notation is also analyzed, together with various fitness functions.
Finally, a new method based on the analysis of Walsh coefficients for finding nonlinear functions is proposed.


\keywords{Boolean Functions, Nonlinearity, Heuristic Optimization, Evolutionary Algorithms}
\end{abstract}

\end{document}
