\chapter{Analiza problema}
Pronalazak visoko nelinearnih Booleovih funkcija je poznat problem na kojemu je provedeno mnogo istraživanja.
Težina problema dolazi od veličine prostora stanja, koji raste super eksponencijalno s porastom broja varijabli.
Općenito vrijedi da za $n$ varijabli Booleova funkcija ima $2^n$ elemenata u tablici istinitosti te postoji $2^{2^n}$ različitih Booleovih funkcija.
Eksponencijalni rast tablice istinitosti također predstavlja implementacijske probleme zbog memorijskih zahtjeva za zapis istoga, ali i vremenskih zahtjeva za računanje.
Osim velikog prostora stanja, problem je i količina visoko nelinearnih Booleovih funkcija, odnosno to što je njihov udio u svim Booleovim funkcijama određenog broja varijabli sve manji s porastom broja varijabli.
Tako primjerice za $4$ varijabli postoji ukupno $65536$ Booleovih funkcija, od čega su $896$ funkcije Bent-funkcije, što čini $1.3\%$ svih funkcija.
Sa $6$ varijabli postoji ukupno $1.84 \times 10^{19}$ različitih Booleovih funkcija, od čega su $5425430528$ funkcije Bent-funkcije, što čini svega $2.94 \times 10^{-8}\%$ svih funkcije.
Za $8$ varijabli ovaj omjer postaje još manji te je samo $8.155 \times 10^{-44}\%$ \cite{DiscoveryOfBent}.
Za broj Bent-funkcija vrijedi ograda dana u izrazu \eqref{eq:bent_num} \cite{CryptographicBooleanFunctions}.
\begin{equation}\label{eq:bent_num}
    \left(2^{\frac{n}{2}}\right)! 2^{2^{\frac{n}{2}}} \leq
    \#bent \leq
    2^{2^{n-1}-\frac{1}{2}\binom{n}{\frac{n}{2}}}.
\end{equation}
Ukupan broj Booleovih funkcija, kao i ograde i stvaran broj Bent-funkcija za određene brojeve varijabli dan je u tablici \ref{tbl:boolean_count}.
\begin{table}[]
\begin{tabular}{c|cccc}
$n$ & donja ograda & broj Bent-funkcija & gornja ograda & broj Booleovih funkcija \\ \hline
$2$ & $8$ & $8$ & $8$ & $8$ \\
$4$ & $384$ & $896$ & $2048$ & $65536$ \\
$6$ & $2^{23.3}$ & $2^{32.3}$ & $2^{38}$ & $2^{64}$ \\
$8$ & $2^{95.6}$ & $2^{106.3}$ & $2^{129.2}$ & $2^{256}$ \\
$10$ & $2^{262.2}$ & $?$ & $2^{612}$ & $2^{1024}$
\end{tabular}
\caption{Ukupan broj Booleovih funkcija i ograde za Bent-funkcije (podatci preuzeti iz \cite{CryptographicBooleanFunctions})}
\label{tbl:boolean_count}
\end{table}

\section{Mjere vrednovanja rješenja}
Uz sve spomenute probleme prilikom pronalaska nelinearnih Booleovih funkcija, dodatan problem javlja se prilikom vrednovanja rješenja.
Korišteni optimizacijski algoritmi temelje se na tome da svakom rješenju dodjeljuju vrijednost u skladu s time koliko je ono dobro ili loše te pomoću te informacije pronalaze nova, potencijalno bolja rješenja.
Ovisno o vrsti mjere vrednovanja, one mogu biti ili mjere odnosno funkcije dobrote ili funkcije kazne.
Funkcije dobrote rješenju pridjeljuju vrijednost na način da ono rješenje koje je bolje dobije veću vrijednost, dok funkcije kazne veću vrijednost dodjeljuju onom rješenju koje je lošije.
Ovisno o tome koja vrsta funkcija vrednovanja se koristi, optimizacijski problem se postavlja kao maksimizacijski u slučaju funkcija dobrote, odnosno minimizacijski u slučaju funkcija kazne.

Najjednostavnija korištena mjera vrednovanja rješenja je upravo iznos nelinearnosti funkcije $N_f$.
Navedena mjera pruža jednostavno tumačenje uspješnosti rješenja.
Problem ovako definirane funkcije dobrote je u tome što velik broj različitih Booleovih funkcija posjeduje jednaku razinu nelinearnosti, zbog čega je postoje velika područja u kojima optimizacijski algoritmi nemaju informaciju o tome napreduju li ka boljem rješenju.

Sljedeća korištena mjera vrednovanja je nadogradnja prethodne na način da se osim nelinearnosti 
Booleove funkcije također vrednuje i iznos drugog po veličini Walshovog koeficijenta.
Ideja ove mjere je iskoristiti svojstvo Walshovih koeficijenata da predstavljaju udaljenosti od pojedinih afinih Booleovih funkcija.
Na taj se način od više funkcija jednake nelinearnosti prednost daje onima čija je udaljenost od sljedeće najbliže afine funkcije maksimalna.
Time je povećana vjerojatnost da povećavanje udaljenosti od trenutno najbliže afine funkcije rezultira i povećanjem nelinearnosti, jer neovisno o tome je li udaljenost od sljedeće najbliže afine funkcije povećana ili smanjena, ona je minimalna s obzirom na do tada pronađena rješenja.
Dodatna motivacija za ovako definiranom mjerom vrednovanja je u tome što je poznato da Bent-funkcije imaju sve Walshove koeficijente jednakog iznosa, što znači da je prilikom traženja Bent funkcija cilj minimizirati sve Walshove koeficijente, gledano po njihovoj apsolutnoj vrijednosti.

Treća korištena mjera vrednovanja dana je izrazom \eqref{eq:cost_function}, koja je korištena u mnogim radovima, poput: \cite{MaximalNonlinearity}, \cite{CryptographicBoolean}, \cite{EvolvingBoolean} i \cite{picek2016new}.
\begin{equation}\label{eq:cost_function}
    cost = \sum_{\vec{w}\in \mathds{B}^n} \abs{\abs{W_f(\vec{w)}} - X}^R
\end{equation}
Navedena mjera je funkcija kazne, koja se temelji na sličnom principu prethodne mjere vrednovanja, samo što se umjesto najveća dva koeficijenta Walshovog spektra koriste svi koeficijenti.
Funkcija posjeduje poželjna svojstva u vidu kažnjavanja velikih vrijednosti Walshovih koeficijenata, uz mogućnost utjecaja na način kažnjavanja kroz parametre $X$ i $R$.
Premda spomenuti parametri predstavljaju mogućnost podešavanja funkcije kazne, oni također predstavljaju i problem s obzirom na to da uvode potrebu za pomnim odabirom istih te ispitivanje njihovog utjecaja na rad optimizacijskih algoritama.
Za potrebe ovog rada korištene su vrijednosti $R=3$ te $X=4$, koje su se pokazale kao najuspješnije u dosadašnjim radovima \cite{EvolvingBoolean}.

\section{Analiza susjedstva}