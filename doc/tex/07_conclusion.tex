\chapter{Zaključak}

U okviru ovog rada implementirano je i testirano nekoliko različitih pristupa za pronalazak Booleovih funkcija visoke nelinearnosti.
Korišteni su sljedeći optimizacijski algoritmi: iterativni algoritam pretraživanja, metoda uspona na vrh, simulirano kaljenje, genetski algoritam i genetsko programiranje te se pokazalo kako je uz korištenje genetskog programiranja moguće pronaći Bent-funkcije u najmanjem broju koraka pretrage.
Također je korišteno nekoliko različitih mjera vrednovanja rješenja, čime se pokazalo kako za većinu algoritama funkcija kazne iz izraza \eqref{eq:cost_function} pronalazi rješenje u najmanjem broju koraka, zahvaljujući najvećoj informiranosti funkcije.
Isto nije bio slučaj kod genetskog programiranja, gdje se kao najuspješnija mjera vrednovanja pokazala funkcija dobrote koja poprima vrijednost nelinearnosti funkcije.
Pored navedenog, predložena je nova funkcija za određivanje susjedstva, odnosno nova funkcija mutacije kod genetskog algoritma, slična onoj predloženoj u radu \cite{millan1997smart}, uz poboljšanja u vidu vremenske složenosti.
Predložena funkcija pokazala se uspješnom te je omogućila pronalazak rješenja u manjem broju koraka pretrage.

Potraga za balansiranom Booleovom funkcijom maksimalne nelinearnosti od $8$ varijabli rezultirala je pronalaskom funkcija nelinearnosti $116$, što je jednako do sada najbolje pronađenim rješenjima, ali ostavlja otvoreno pitanje o postojanju balansirane Booleove funkcije nelinearnosti $118$. 

Korištenjem genetskog programiranja ustanovljena je pravilnost prilikom generiranja rješenja uz korištenje postojećih Bent-funkcija.
Točnije, zamijećeno je da jednom kad je pronađeno rješenje koje koristi Bent-funkciju manjeg broja varijabli, moguće je korištenu Bent-funkciju zamijeniti bilo kojom drugom Bent-funkcijom jednakog broja varijabli čime se ponovno ostvaruje Bent-funkcija.
Navedeno svojstvo samo je eksperimentalno potvrđeno te je dokazivanje i daljnje korištenje istoga ostavljeno za daljnji rad.