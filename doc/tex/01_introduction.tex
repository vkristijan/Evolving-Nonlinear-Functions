\chapter{Uvod}
Kriptografija je grana računarske znanosti, bez čijih dostignuća bi današnji svakodnevni život izgledao nezamislivo drugačije.
Većina se ljudi svakodnevno služi kriptografijom, makar toga nisu nužno ni svjesni.
Bilo to korištenjem online bankarstva, online trgovine ili jednostavno bilo kojeg online servisa temeljenog na \textit{HTTPS} protokolu.
Navedeni primjeri često se služe asimetričnim kriptografskim algoritmima kako bi klijent i poslužitelj razmijenili ključeve koje potom koriste u simetričnim kriptografskim algoritmima kojima se enkriptiraju njihove poruke.

Booleove funkcije predstavljaju jedan od sastavnih elemenata kriptografskih algoritama, poput na primjer S-kutija kod algoritma \textit{DES} i \textit{AES} \cite{daemen1999aes}.
Kako bi se postigla što veća sigurnost takvih sustava, potrebno je pomno odabrati korištene funkcije, ovisno o određenim svojstvima koje one posjeduju.
Jedan od najpoznatijih napada je napad linearnom kriptoanalizom, prvi puta opisan u \cite{golic1994linear}.
Kako bi se postigla veća otpornost na tu vrstu napada, od posebnog je značaja svojstvo nelinearnosti Booleove funkcije.
Pored toga, za primjenu u kriptografskim algoritmima dodatno je važno i svojstvo balansiranosti.
Osim navedenog, važna su i brojna druga svojstva poput svojstva korelacijske otpornosti \engl{correlation immunity}, autokorelacije \engl{autocorrelation}, algebarski stupanj \engl{algebraic degree}, algebarske otpornosti \engl{algebraic immunity} i drugih opisanih u \cite{CryptographicBooleanFunctions}.

Primarni fokus ovog rada je pronalazak Booleovih funkcija maksimalne nelinearnosti, kao i balansiranih Booleovih funkcija maksimalne nelinearnosti, pritom ne razmatrajući ostala svojstva.
Posebna pozornost posvećena je balansiranim Booleovim funkcijama sa $8$ varijabli, koje se koriste u algoritmu \textit{Achterbahn} \cite{gammel2005achterbahn}, kao i u algoritmu \textit{RAKAPOSHI} \cite{cid2009rakaposhi}, za koje je pitanje maksimalne moguće nelinearnosti još uvijek otvoreno.
Naime, najniža gornja granica nelinearnosti navedenih funkcija iznosi $118$, dok je najviša do sad pronađena nelinearnost u iznosu $116$.

\section{Booleove funkcije i njihova svojstva}
Funkcija $f$ u općenitom je smislu definirana kao preslikavanje člana jednog skupa, koji se naziva domena u točno određeni član drugog skupa, koji se naziva kodomena.

Ako su domena i kodomena skup realnih brojeva $\mathds{R}$, funkcija $f : \mathds{R}^n \rightarrow \mathds{R}$ zove se \textbf{realna funkcija}.

Realna funkcija $f$ je \textbf{linearna funkcija} ako je oblika:
\begin{equation}
    f(\vec{x}) = a_1x_1 + a_2x_2 + \dots + a_nx_n,
\end{equation}
gdje su $a_1, a_2, \dots, a_n \in \mathds{R}$, $x_1, x_2, \dots, x_n \in \mathds{R}$ te $f(\vec{x}) \in \mathds{R}$.

Realna funkcija je \textbf{afina funkcija} ako je oblika:
\begin{equation}
    f(\vec{x}) = a_0 + a_1x_1 + a_2x_2 + \dots + a_nx_n,
\end{equation}
gdje su $a_0, a_1, a_2, \dots, a_n \in \mathds{R}$, $x_1, x_2, \dots, x_n \in \mathds{R}$ te $f(\vec{x}) \in \mathds{R}$.

\textbf{Booleova funkcija} je funkcija čija su domena i kodomena iz skupa $\mathds{B}$, što je skup elemenata $\mathds{B} = \{0, 1\}$.

Slično kao i za realne funkcije, moguće je definirati linearnu i afinu Boolevou funkciju.
Pritom se umjesto operacija zbrajanja koristi operacija logičkog isključivo ili \engl{exclusive or}, odnosno \textit{XOR}, dok se umjesto množenja koristi operacija konjunkcije, odnosno \textit{AND}.

Booleova funkcija je linearna, ako je oblika:
\begin{equation}
    f(\vec{x}) = a_1x_1 \oplus a_2x_2 \oplus \dots \oplus a_nx_n,
\end{equation}
gdje su $a_1, a_2, \dots, a_n \in \mathds{B}$, $x_1, x_2, \dots, x_n \in \mathds{B}$ te $f(\vec{x}) \in \mathds{B}$.

Booleova funkcija je afina, ako je oblika:
\begin{equation}
    f(\vec{x}) = a_0 \oplus a_1x_1 \oplus a_2x_2 \oplus \dots \oplus a_nx_n,
\end{equation}
gdje su $a_0, a_1, a_2, \dots, a_n \in \mathds{B}$, $x_1, x_2, \dots, x_n \in \mathds{B}$ te $f(\vec{x}) \in \mathds{B}$.

Booleovu funkciju moguće je jednoznačno definirati \textbf{tablicom istinitosti}, što je vektor vrijednosti funkcije za sve moguće kombinacije varijabli, poredanih prema leksikografskom poretku vrijednosti ulaznih varijabli.

Booleova funkcija je \textbf{balansirana} ako u tablici istinitosti sadrži jednak broj vrijednosti $0$ i vrijednosti $1$.

Za dvije Booleove funkcije, definirana je \textbf{Hammingova udaljenost} kao broj bitova po kojima se njihove tablice istinitosti međusobno razlikuju.

\textbf{Nelinearnost} Booleove funkcije definirana je kao minimalna Hammingova udaljenost od svih afinih Booleovih funkcija.

Booleova funkcija $f$ je \textbf{Bent-funkcija} ako za dani broj varijabli posjeduje svojstvo maksimalne nelinearnosti te je jednako udaljena od svih afinih Booleovih funkcija.
Bent funkcija postoje samo za paran broj varijabli, ali postoji i skupina semi-bent-funkcija definirana za neparan broj varijabli, što je opisano u knjizi \cite{CryptographicBooleanFunctions}.

\section {Walshov spektar}
Walshova transformacija Booleove funkcije $f$ definirana je kao:
\begin{equation}\label{eq:walsh transform}
    W_f(\vec{w}) = \sum_{\vec{x} \in \mathds{B}^n}f(\vec{x})(-1)^{\vec{w}\vec{x}},
\end{equation}
čime je definiran Walshov koeficijent $W_f(\vec{w})$.
Skup Walshovih koeficijenata za sve $\vec{w} \in \mathds{B}^n$, zove se \textbf{Walshov spektar} Booleove funkcije \cite{CryptographicBooleanFunctions}.

Svaki Walshov koeficijent predstavlja neočekivanu udaljenost \engl{unexpected distance} \cite{MeasuringBoolean} Booleove funkcije $f$ od pripadne linearne Booleove funkcije.
Za svaki par Booleovih funkcija, očekivana vrijednost Hammingove udaljenosti iznosi pola od ukupnog broja bitova u tablici istinitosti, odnosno $2^{n-1}$.
Neočekivana udaljenost definirana je kao razlika stvarne Hammingove udaljenosti dviju funkcija i njihove očekivane udaljenosti.

Na primjeru funkcije čija je tablica istinitosti $00111010$, Walshov koeficijent za $\vec{v} = 010$, odgovarati će neočekivanoj udaljenosti od linearne funkcije $f(\vec{x}) = x_1$, čija je tablica istinitosti jednaka $00110011$.
Kako je riječ o funkcijama $3$ varijable, očekivana udaljenost iznosi $2^{3-1} = 4$, dok je stvarna Hammingova udaljenost $2$.
Razlika te dvije vrijednosti daje $-2$, što je vrijednost neočekivane udaljenosti dane funkcije, kao i pripadni Walshov koeficijent.

Walshov spektar moguće je izračunati direktno korištenjem izraza \ref{eq:walsh transform}, što je moguće izvesti u vremenskoj složenosti $\mathcal{O}(n^2)$.
Isto je moguće izračunati u vremenskoj složenosti $\mathcal{O}(n\log n)$ korištenjem brze Walshove transformacije \engl{fast Walsh transform}.
Taj se postupak provodi uzastopnom primjenom operatora koji za par brojeva $(a, b)$, računa novi par $(a', b')$ prema izrazu:
\begin{equation}\label{eq:fwt}
    (a', b') = (a+b, a-b).
\end{equation}
\begin{figure}[ht!] 
    \centering
    \includegraphics[width=.6\textwidth]{img/fwt_example}
    \captionsetup{justification=centering}
    \caption{Primjer izračuna brze Walshove transformacije nad Booleovom funkcijom čija je tablica istinitosti $00111010$}
    \label{fig:fwt_example}
\end{figure}
Na slici \ref{fig:fwt_example} prikazan je cjelokupni postupak izračuna brze Walshove transformacije za prethodno prikazanu funkciju.
Postupak se provodi u $\log n$ koraka, a svaki je korak prikazan jednim stupcem u slici.
Stupci na slici međusobno su povezani linijama koje prikazuju iz kojih brojeva je izračunat broj u pojedinom stupcu.
Točnije, svaki broj je povezan sa točno $2$ broja iz prethodnog stupca, što ukazuje na to da je on dobiven zbrojem ili razlikom ta dva broja, u skladu s operatorom \ref{eq:fwt}.
Prvi stupac prikazuje Booleovu funkciju nad kojom se računa brza Walshova transformacija.
U prvom koraku algoritma, tablica istinitosti se dijeli na grupe veličine $2$, te se na svaku skupinu primjenjuje operator \ref{eq:fwt}, čime se dobije rezultat prikazan u drugom stupcu slike.
Sljedeći korak koristi grupe veličine $4$ te se operator \ref{eq:fwt} primjenjuje redom na sve parove međusobno udaljene za $2$.
Općenito vrijedi da se u $n$-tom koraku brojevi grupiraju u grupe veličine $2^n$, a operator \ref{eq:fwt} se primjenjuje na one brojeve unutar grupe koji se nalaze na međusobnoj udaljenosti $2^{n-1}$.
Postupak se ponavlja sve dok u zadnjem koraku ne preostane samo jedna grupa koja uključuje sve brojeve.

Iz opisanog postupka, kao i iz slike \ref{fig:fwt_example} moguće je primjetiti važno svojstvo, točnije kako svaki bit tablice istinitosti utječe na iznos pojedinog Walshovog koeficijenta. 
Navedeno svojstvo prikazano je podebljanom linijom na slici, čime su prikazani svi bitovi koji su uključeni u izračun trećeg Walshovog koeficijenta.
Također se primjećuje kako vrijedi i obrat, odnosno da svaki pojedini bit utječe na sve vrijednosti Walshovog spektra.

Primjećuje se kako opisani postupak rezultira ispravnim Walshovim koeficijentima za cijeli Walshov spektar booleove funkcje, osim za prvi koeficijent.
Dobiven prvi koeficijent odgovara Hammingovoj težini Booleove funkcije, odnosno broju jedinica u njezinoj tablici istinitosti.
Ova činjenica predstavlja dodatno kompliciranje postupka izračuna vrijednosti željenih svojstava, s obzirom na to da je prvi broj potrebno obrađivati na drugi način od ostatka spektra.
U radu \cite{MeasuringBoolean} taj je problem riješen na način da se na vrijednosti tablice istinitosti primijeni sljedeća transformacija:
\begin{equation}
    x' = -1^x.
\end{equation}
Na taj se način skup ${0, 1}$ preslikava u skup ${1, -1}$.
Postupak izračuna brze Walshove transformacije uz korištenje navedene transformacije ulaznih bitova prikazan je na slici \ref{fig:transformed_fwt_example}.
\begin{figure}[ht!] 
    \centering
    \includegraphics[width=.6\textwidth]{img/transformed_fwt_example}
    \captionsetup{justification=centering}
    \caption{Primjer izračuna brze Walshove transformacije nad Booleovom funkcijom čija je tablica istinitosti $00111010$ uz primjenu transformacije bitova}
    \label{fig:transformed_fwt_example}
\end{figure}
Novo dobiveni rezultati razlikuju se od prethodnih po predznaku i magnitudi.
Usprkos tome, rezultati u prikladniji zbog prvog koeficijenta koji po svojstvima odgovara ostalima.

Pomoću Walshovog spektra računaju se ostala svojstva Booleove funkcije.
Funkcija je balansirana ako je njezin prvi Walshov koeficijent jednak nuli \cite{MaximalNonlinearity}:
\begin{equation}
    W_f(\vec{0}) = 0.
\end{equation} 

Nelinearnost $N_f$ Booleove funkcije $f$ određena je apsolutno najvećim Walshovim koeficijentom prema izrazu \cite{MaximalNonlinearity}:
\begin{equation}
    N_f = 2^{n-1} - \frac{1}{2}\max_{\vec{w} \in \mathds{B}^n} \abs{W_f(\vec{w})}.
\end{equation}

Za svaku Booleovu funkciju $f$ sa $n$ varijabli vrijedi Parsevalova jednakost koja glasi:
\begin{equation}
    \sum_{\vec{w}\in \mathds{B}^n} \left( W_f(\vec{w}) \right)^2 = 2^{2n}.
\end{equation} 
Iz toga slijedi da nelinearnost $N_f$ Bent-funkcije sa $n$ varijabli iznosi:
\begin{equation}
    N_f = 2^{n-1} - 2^{\frac{n}{2}-1}.
\end{equation}