\chapter{Uvod}
Kriptografija je grana računarske znanosti, bez čijih dostignuća bi današnji svakodnevni život izgledao nezamislivo drugačije.
Većina se ljudi svakodnevno služi kriptografijom, makar toga nisu nužno ni svjesni.
Bilo to korištenjem online bankarstva, online trgovine ili jednostavno bilo kojeg online servisa temeljenog na \textit{HTTPS} protokolu.
Navedeni primjeri često se služe asimetričnim kriptografskim algoritmima kako bi klijent i poslužitelj razmijenili ključeve koje potom koriste u simetričnim kriptografskim algoritmima kojima se enkriptiraju njihove poruke.

Booleove funkcije predstavljaju jedan od sastavnih elemenata kriptografskih algoritama, poput na primjer S-kutija kod algoritma \textit{DES} i \textit{AES} \cite{daemen1999aes}.
Kako bi se postigla što veća sigurnost takvih sustava, potrebno je pomno odabrati korištene funkcije, ovisno o određenim svojstvima koje one posjeduju.
Jedan od najpoznatijih napada je napad linearnom kriptoanalizom, prvi puta opisan u \cite{golic1994linear}.
Kako bi se postigla veća otpornost na tu vrstu napada, od posebnog je značaja svojstvo nelinearnosti Booleove funkcije.
Pored toga, za primjenu u kriptografskim algoritmima dodatno je važno i svojstvo balansiranosti.
Osim navedenog, važna su i brojna druga svojstva poput svojstva korelacijske otpornosti \engl{correlation immunity}, autokorelacije \engl{autocorrelation}, algebarski stupanj \engl{algebraic degree}, algebarske otpornosti \engl{algebraic immunity} i drugih opisanih u \cite{CryptographicBooleanFunctions}.

Primarni fokus ovog rada je pronalazak Booleovih funkcija maksimalne nelinearnosti, kao i balansiranih Booleovih funkcija maksimalne nelinearnosti, pritom ne razmatrajući ostala svojstva.
Posebna pozornost posvećena je balansiranim Booleovim funkcijama sa $8$ varijabli, koje se koriste u algoritmu \textit{Achterbahn} \cite{gammel2005achterbahn}, kao i u algoritmu \textit{RAKAPOSHI} \cite{cid2009rakaposhi}, za koje je pitanje maksimalne moguće nelinearnosti još uvijek otvoreno.
Naime, najniža gornja granica nelinearnosti navedenih funkcija iznosi $118$, dok je najviša do sad pronađena nelinearnost u iznosu $116$.