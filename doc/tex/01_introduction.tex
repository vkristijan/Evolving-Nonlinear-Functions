\chapter{Uvod}
Kriptografija je grana računarske znanosti, bez čijih dostignuća bi današnji svakodnevni život izgledao nezamislivo drugačije.
Većina se ljudi svakodnevno služi kriptografijom, premda toga nisu nužno ni svjesni.
Bilo to korištenjem internet bankarstva, internet trgovine ili jednostavno bilo kojeg internet servisa temeljenog na protokolu \textit{HTTPS}.
Navedeni primjeri često se služe asimetričnim kriptografskim algoritmima kako bi klijent i poslužitelj razmijenili ključeve koje potom koriste u simetričnim kriptografskim algoritmima kojima se enkriptiraju njihove poruke.

Booleove funkcije predstavljaju jedan od sastavnih elemenata kriptografskih algoritama, poput S-kutija kod algoritma \textit{DES} i \textit{AES} \cite{daemen1999aes}.
Kako bi se postigla što veća sigurnost takvih sustava, potrebno je pomno odabrati korištene funkcije, ovisno o određenim svojstvima koje one posjeduju.
Jedan od najpoznatijih napada je napad linearnom kriptoanalizom, prvi puta opisan u \cite{golic1994linear}.
Kako bi se postigla veća otpornost na tu vrstu napada, od posebnog je značaja svojstvo nelinearnosti Booleove funkcije.
Pored toga, za primjenu u kriptografskim algoritmima dodatno je važno i svojstvo balansiranosti.
Osim navedenog, važna su i brojna druga svojstva poput svojstva korelacijske otpornosti \engl{correlation immunity}, autokorelacije \engl{autocorrelation}, algebarski stupanj \engl{algebraic degree}, algebarske otpornosti \engl{algebraic immunity} i drugih opisanih u \cite{CryptographicBooleanFunctions}.

Primarni fokus ovog rada je pronalazak Booleovih funkcija maksimalne nelinearnosti, kao i balansiranih Booleovih funkcija maksimalne nelinearnosti, pritom ne razmatrajući ostala svojstva.
Posebna pozornost posvećena je balansiranim Booleovim funkcijama s $8$ varijabli, koje se koriste u algoritmu \textit{Achterbahn} \cite{gammel2005achterbahn}, kao i u algoritmu \textit{RAKAPOSHI} \cite{cid2009rakaposhi}, za koje je pitanje maksimalne moguće nelinearnosti još uvijek otvoreno.
Naime, najniža gornja granica nelinearnosti navedenih funkcija iznosi $118$, dok je najviša do sad pronađena nelinearnost u iznosu $116$.

Nastavak rada podijeljen je u $6$ poglavlja.
U drugom poglavlju objašnjen je pojam Booleove funkcije, kao i neka od njezinih svojstava korištena u daljnjem radu.
Prikazani su i razni načini zapisa funkcije, kao i načini izračuna opisanih svojstava.
Treće poglavlje predstavlja pregled područja i opisuje dosadašnje pristupe pronalasku visoko nelinearnih Booleovih funkcija.
U četvrtom poglavlju ukratko su objašnjene osnove optimizacijskih algoritama koji su korišteni, a to su iterativni algoritam pretrage, metoda uspona na vrh, simulirano kaljenje, genetski algoritam i genetsko programiranje.
Detaljniji opis problema dan je u petom poglavlju.
U istom poglavlju su navedene i objašnjene korištene mjere vrednovanja rješenja te je provedena analiza susjedstva.
Implementacija optimizacijskih algoritama, kao i rezultati te analiza i usporedba rezultata prikazana je u šestom poglavlju. 